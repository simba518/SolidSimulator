\documentclass[9pt,twocolumn]{extarticle}

\usepackage[hmargin=0.5in,tmargin=0.5in]{geometry}
\usepackage{amsmath,amssymb}
\usepackage{times}
\usepackage{graphicx}
\usepackage{subfigure}

\usepackage{cleveref}
\usepackage{color}
\newcommand{\TODO}[1]{\textcolor{red}{#1}}

\newcommand{\FPP}[2]{\frac{\partial #1}{\partial #2}}
\newcommand{\argmin}{\operatornamewithlimits{arg\ min}}
\author{Siwang Li}

\title{Material}

%% document begin here
\begin{document}
\maketitle

\setlength{\parskip}{0.5ex}

\section{Material}
\begin{center}
  \begin{tabular}{ | l | l | l|l|}
    \hline
	&Density $kg/m^3$ & Young's $10^9N/m^2$ & Possion's  \\ \hline
	rubbon&$[800,1500]$& $[0.01,0.1]$ & $[0.25,0.5]$ \\ \hline
  \end{tabular}
\end{center}
The gravity is $9.8m/s^2$.

%% references
\begin{thebibliography}{99}
\bibitem{density} http://www.simetric.co.uk/si\_materials.htm
\bibitem{young} http://www.engineeringtoolbox.com/young-modulus-d\_417.html
\end{thebibliography}

\end{document}
